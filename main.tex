\documentclass{article}
\usepackage{graphicx} % Required for inserting images
\usepackage[a4paper, margin=1in]{geometry}
% math
\usepackage{amsmath, amssymb, amsthm, stmaryrd, bbold}
% plot
\usepackage{tikz, pgfplots,tikz-cd}
\usetikzlibrary{calc, decorations.markings, decorations.pathmorphing,shapes}
% form
\usepackage{diagbox}
% link
\usepackage[colorlinks, citecolor=blue, linkcolor=blue, urlcolor=blue, breaklinks=true]{hyperref}
% others
\usepackage{mdframed}
\usepackage{natbib}
\allowdisplaybreaks[4]
\pgfplotsset{compat=1.18}

% --- FONT STUFF ---
\usepackage{newpxtext}

\usepackage[T1]{fontenc}

\definecolor{CUPurple}{RGB}{117,15,109}
\definecolor{CULightPurple}{RGB}{172, 111, 167}
\definecolor{CULightPurple2}{RGB}{214, 183, 211}

%%Blues
\definecolor{YaleBlue} {HTML}{00356b}
%\definecolor{YaleMidBlue} {HTML}{286dc0}
%\definecolor{YaleLightBlue} {HTML}{63aaff}

\definecolor{YaleMidBlue} {HTML}{286dc0}
\definecolor{YaleLightBlue} {HTML}{b0d4ff}
\definecolor{YaleMidGrey} {HTML}{e0dcda}
\definecolor{YaleLightGrey} {HTML}{eeeeee}

%%Green
\definecolor{ao}{rgb}{117, 15, 109}

%%Greys
\definecolor{YaleBlack} {HTML}{222222}
\definecolor{YaleDarkGrey} {HTML}{4a4a4a}
%\definecolor{YaleMidGrey} {HTML}{978d85}
%\definecolor{YaleLightGrey} {HTML}{dddddd}
\definecolor{YaleWhite} {HTML}{f9f9f9}
\definecolor{YaleLightGreen}{HTML}{AFB896}
%%Accents
\definecolor{YaleGreen} {HTML}{5f712d}
\definecolor{YaleOrange} {HTML}{bd5319}

\newcommand{\spd}[1]{(#1+1)d}
\newcommand{\mb}[1]{\mathbf{#1}}

%% Note
\newcommand{\XY}[1]{\textcolor{YaleMidBlue}{[Xinping: #1]}}
\newcommand{\XYY}[1]{\textcolor{YaleMidBlue}{[#1]}}
\newcommand{\CQ}[1]{\textcolor{YaleMidGrey}{[Chenqi: #1]}} 

% --- Theorem
\theoremstyle{definition}
\newtheorem{theorem}{Theorem}[section]
\newtheorem{corollary}[theorem]{Corollary}
\newtheorem{prop}[theorem]{Proposition}
\newtheorem{definition}[theorem]{Definition}
\newtheorem{proposition}[theorem]{Proposition}
\newtheorem{lem}[theorem]{Lemma}
\theoremstyle{remark}
\newtheorem{remark}{Remark}
\newtheorem{convention}{Convention}
\newtheorem{conjecture}{Conjecture}
\newtheorem{example}{Example}
\newtheorem{question}{Question}

% --- Macros ---

%% Fields
\newcommand{\Z}{\mathbb{Z}}
\newcommand{\R}{\mathbb{R}}
\newcommand{\C}{\mathbb{C}}
\newcommand{\bbH}{\mathbb{H}}
\newcommand{\Q}{\mathbb{Q}}
\newcommand{\M}{\mathbb{M}}
\newcommand{\N}{\mathbb{N}}
\newcommand{\ii}{\mathrm{i}}
\newcommand{\jj}{\mathrm{j}}
\newcommand{\kk}{\mathrm{k}}
\newcommand{\ee}{\mathrm{e}}

%% Category cal
\newcommand{\cA}{ {\cal A} } 
\newcommand{\cB}{ {\cal B} }
\newcommand{\cC}{ {\cal C} } 
\newcommand{\cD}{ {\cal D} } 
\newcommand{\cE}{ {\cal E} } 
\newcommand{\cF}{ {\cal F} } 
\newcommand{\cG}{ {\cal G} } 
\newcommand{\cH}{ {\cal H} } 
\newcommand{\cI}{{\cal I}}
\newcommand{\cJ}{{\cal J}}
\newcommand{\cK}{ {\cal K} } 
\newcommand{\cL}{ {\cal L} } 
\newcommand{\cM}{ {\cal M} } 
\newcommand{\cN}{ {\cal N} } 
\newcommand{\cO}{\mathcal{O}} 
\newcommand{\cP}{ {\cal P} } 
\newcommand{\cQ}{ {\cal Q} } 
\newcommand{\cR}{ {\cal R} } 
\newcommand{\cS}{ {\cal S} } 
\newcommand{\cT}{ {\cal T} } 
\newcommand{\cU}{ {\cal U} } 
\newcommand{\cV}{ {\cal V} } 
\newcommand{\cW}{ {\cal W} } 
\newcommand{\cX}{ {\cal X} } 
\newcommand{\cY}{ {\cal Y} } 
\newcommand{\cZ}{ {\cal Z} }

%% eucal
\newcommand{\eA}{\eucal{A}}
\newcommand{\eB}{\eucal{B}}
\newcommand{\eC}{\eucal{C}}
\newcommand{\eD}{\eucal{D}}
\newcommand{\eE}{\eucal{E}}
\newcommand{\eF}{\eucal{F}}
\newcommand{\eG}{\eucal{G}}
\newcommand{\eH}{\eucal{H}}
\newcommand{\eI}{\eucal{I}}
\newcommand{\eJ}{\eucal{J}}
\newcommand{\eK}{\eucal{K}}
\newcommand{\eL}{\eucal{L}}
\newcommand{\eM}{\eucal{M}}
\newcommand{\eN}{\eucal{N}}
\newcommand{\eO}{\eucal{O}}
\newcommand{\eP}{\eucal{P}}
\newcommand{\eQ}{\eucal{Q}}
\newcommand{\eR}{\eucal{R}}
\newcommand{\eS}{\eucal{S}}
\newcommand{\eT}{\eucal{T}}
\newcommand{\eU}{\eucal{U}}
\newcommand{\eV}{\eucal{V}}
\newcommand{\eW}{\eucal{W}}
\newcommand{\eX}{\eucal{X}}
\newcommand{\eY}{\eucal{Y}}
\newcommand{\eZ}{\eucal{Z}}

%% sf
\newcommand{\sA}{ {\mathsf A} } 
\newcommand{\sB}{ {\mathsf B} }
\newcommand{\sC}{ {\mathsf C} } 
\newcommand{\sD}{ {\mathsf D} } 
\newcommand{\sE}{ {\mathsf E} } 
\newcommand{\sF}{ {\mathsf F} } 
\newcommand{\sG}{ {\mathsf G} } 
\newcommand{\sH}{ {\mathsf H} } 
\newcommand{\sI}{{\mathsf I}}
\newcommand{\sJ}{{\mathsf J}}
\newcommand{\sK}{ {\mathsf K} } 
\newcommand{\sL}{ {\mathsf L} } 
\newcommand{\sM}{ {\mathsf M} } 
\newcommand{\sN}{ {\mathsf N} } 
\newcommand{\sO}{ {\mathsf O} } 
\newcommand{\sP}{ {\mathsf P} } 
\newcommand{\sQ}{ {\mathsf Q} } 
\newcommand{\sR}{ {\mathsf R} } 
\newcommand{\sS}{ {\mathsf S} } 
\newcommand{\sT}{ {\mathsf T} } 
\newcommand{\sU}{ {\mathsf U} } 
\newcommand{\sV}{ {\mathsf V} } 
\newcommand{\sW}{ {\mathsf W} } 
\newcommand{\sX}{ {\mathsf X} } 
\newcommand{\sY}{ {\mathsf Y} } 
\newcommand{\sZ}{ {\mathsf Z} }

%% frak
\newcommand{\fC}{\mathfrak{C}}
\newcommand{\fD}{\mathfrak{D}}
\newcommand{\fA}{\mathfrak{A}}
\newcommand{\fB}{\mathfrak{B}}
\newcommand{\fS}{\mathfrak{S}}
\newcommand{\fZ}{\mathfrak{Z}}

%% Category bf
\newcommand{\BB}{\mathbf{B}}
\newcommand{\Bm}{\mathbf{m}} 
\newcommand{\Bn}{\mathbf{n}} 
\newcommand{\Br}{\mathbf{r}} 
\newcommand{\Bs}{\mathbf{s}} 
\newcommand{\Bt}{\mathbf{t}} 
\newcommand{\Bu}{\mathbf{u}} 
\newcommand{\Bv}{\mathbf{v}} 
\newcommand{\Bw}{\mathbf{w}} 
\newcommand{\Bp}{\mathbf{p}}
\newcommand{\BI}{\mathbf{I}}
\newcommand{\Ve}{\mathsf{Vec}}
\newcommand{\sVec}{\mathsf{sVec}}
\newcommand{\Hilb}{\mathsf{Hilb}}
\newcommand{\sHilb}{\mathsf{sHilb}}
\newcommand{\ssHilb}{\underline{\mathbf{sHilb}}}
\newcommand{\Grp}{\mathbf{Grp}}
\newcommand{\Ab}{\mathbf{Ab}}
\newcommand{\Is}{\mathbf{Is}}
\newcommand{\one}{\mathbf{1}}
\newcommand{\zero}{\mathbf{0}}

%% Category rm
\newcommand{\Tr}{\mathrm{Tr}}
\newcommand{\tTr}{\mathrm{tTr}}
\newcommand{\ev}{\mathrm{ev}}
\newcommand{\coev}{\mathrm{coev}}
\newcommand{\Ob}{\mathop\mathrm{Ob}}
\newcommand{\Hom}{\mathrm{Hom}}
\newcommand{\End}{\mathrm{End}}
\newcommand{\categoricalEnd}{{\mathsf{End}}}
\newcommand{\Aut}{\mathrm{Aut}}
\newcommand{\Fun}{\mathsf{Fun}}
\newcommand{\rev}{\mathrm{rev}}
\newcommand{\Pen}{\mathrm{Pen}}
\newcommand{\Rep}{\mathsf{Rep}}
\newcommand{\sRep}{\mathop\mathrm{sRep}}
\newcommand{\Mod}{\mathop\mathrm{Mod}}
\newcommand{\RMod}{\mathrm{RMod}}
\newcommand{\LMod}{\mathrm{LMod}}
\newcommand{\BMod}{\mathrm{BMod}}
\newcommand{\Free}{\mathrm{Free}}
\newcommand{\id}{\mathrm{id}}
\newcommand{\im}{\mathrm{im}}
\newcommand{\TY}{\mathrm{TY}}
\newcommand{\Irr}{\mathrm{Irr}}
\newcommand{\Idem}{\mathrm{Idem}}
\newcommand{\B}{\mathrm{B}}
\newcommand{\chargef}{\mathsf{fgt}}


%% Mathematical operators
\newcommand{\dd}{\mathrm{d}}
\newcommand{\xt}{\times}
\newcommand{\ot}[1][]{\underset{#1}{\otimes}}
\newcommand{\op}[1][]{\underset{#1}{\oplus}}
\newcommand{\oop}{\bigoplus}
\newcommand{\oot}{\bigotimes}
\newcommand{\rt}[1][]{\underset{#1}{\triangleright}}
\newcommand{\lt}[1][]{\underset{#1}{\triangleleft}}
\newcommand{\bt}[1][]{\underset{#1}{\boxtimes}}
\newcommand{\ar}{\rightarrow}
\newcommand{\Ar}{\Rightarrow}
\newcommand{\inj}{\hookrightarrow}
\newcommand{\sur}{\twoheadrightarrow}
\newcommand{\cen}[2]{{#1}^\mathrm{cen}_{#2}}
\newcommand{\Cp}{\Sigma}
\newcommand{\cond}{\mathrel{\,\hspace{.75ex}\joinrel\rhook\joinrel\hspace{-.75ex}\joinrel\rightarrow}}
\newcommand{\Cond}{\mathrel{\,\joinrel\raisebox{0.25ex}{$\rhook$}\joinrel\hspace{-1ex}\joinrel\Rightarrow}}

%% Brackets
\newcommand{\la}{\langle} 
\newcommand{\ra}{\rangle}
\newcommand{\sym}{\mathrm{sym}}
\newcommand{\scr}{\mathrm{scr}}
\newcommand{\bl}[1]{{\hat{#1}}}
\newcommand{\fgt}{\mathrm{fgt}}
\newcommand{\spann}{\mathrm{span}}

%% Tikzs
\tikzstyle string=[thin,postaction={decorate},decoration={markings,
    mark=at position 0.5*\pgfdecoratedpathlength+.5*4pt with {\arrow[scale=1]{stealth}}}]
\newcommand{\diagram}[2]{
\begin{tikzpicture}[baseline=(current bounding box),scale=#1]
#2
\end{tikzpicture}
}

% HW
\newcommand{\pt}[1]{{\color{red}(#1')}}
\newcommand{\der}[2]{\frac{\mathrm{d} #1}{\mathrm{d} #2}}
\newcommand{\pd}[2]{\frac{\partial #1}{\partial #2}}

% *** quiver ***
% A package for drawing commutative diagrams exported from https://q.uiver.app.
%
% This package is currently a wrapper around the `tikz-cd` package, importing necessary TikZ
% libraries, and defining a new TikZ style for curves of a fixed height.
%
% Version: 1.5.3
% Authors:
% - varkor (https://github.com/varkor)
% - AndréC (https://tex.stackexchange.com/users/138900/andr%C3%A9c)

\NeedsTeXFormat{LaTeX2e}
\ProvidesPackage{quiver}[2021/01/11 quiver]

% `tikz-cd` is necessary to draw commutative diagrams.
\RequirePackage{tikz-cd}
% `amssymb` is necessary for `\lrcorner` and `\ulcorner`.
\RequirePackage{amssymb}
% `calc` is necessary to draw curved arrows.
\usetikzlibrary{calc}
% `pathmorphing` is necessary to draw squiggly arrows.
\usetikzlibrary{decorations.pathmorphing}

% A TikZ style for curved arrows of a fixed height, due to AndréC.
\tikzset{curve/.style={settings={#1},to path={(\tikztostart)
    .. controls ($(\tikztostart)!\pv{pos}!(\tikztotarget)!\pv{height}!270:(\tikztotarget)$)
    and ($(\tikztostart)!1-\pv{pos}!(\tikztotarget)!\pv{height}!270:(\tikztotarget)$)
    .. (\tikztotarget)\tikztonodes}},
    settings/.code={\tikzset{quiver/.cd,#1}
        \def\pv##1{\pgfkeysvalueof{/tikz/quiver/##1}}},
    quiver/.cd,pos/.initial=0.35,height/.initial=0}

% TikZ arrowhead/tail styles.
\tikzset{tail reversed/.code={\pgfsetarrowsstart{tikzcd to}}}
\tikzset{2tail/.code={\pgfsetarrowsstart{Implies[reversed]}}}
\tikzset{2tail reversed/.code={\pgfsetarrowsstart{Implies}}}
% TikZ arrow styles.
\tikzset{no body/.style={/tikz/dash pattern=on 0 off 1mm}}

\endinput
\title{From conformal bootstrap to vertex operator algebra}
\author{Chenqi Meng}
\date{June 2025}

\begin{document}

\maketitle


\begin{abstract}
    Vertex operator algebra (VOA) clarifies which part of the CFT is \emph{definition} and which part of it is \emph{derivation}.
    This note reviews conformal field theory algebraically. We avoid the formal introduction of VOA, but always bear in mind that there is some underlying abstract definition of the CFT. 
\end{abstract}
\tableofcontents
\section{Basic ingredients}
\subsection{Kinematical data}
For closed string CFT, one considers dynamics on $S^1$. 

Usually, the Hilbert space on $S^1$ is the Lagrangian algebra in $\frak{Z}(\mathsf{Mod}_{\cV_0})\cong \mathsf{Mod}_{\cV_0}\bt \overline{\mathsf{Mod}_{\cV_0}}$:
\[
\cH = \bigoplus_{h,\bar{h}}N_{h,\bar{h}}\cV_{h}\otimes\overline{\cV}_{\bar{h}}.
\]
Each $\cV_{h}$($\overline{\cV}_{\bar{h}}$) is naturally $\Z$-graded:
\[
\cV_h = \bigoplus_{n\in \Z} V_{h+n}.
\]
The ``ground state'' corresponds to a special vector $\one \otimes\overline{\one} \in V_{0}\otimes \overline{V}_0\subset \cV_{0}\otimes \overline{\cV}_0$.
Here $\cV_0$ is the corresponding VOA, and each $\cV_h$ is a irreducible module of $\cV_0$.

An VOA is the local symmetry of a chiral CFT, which describes the canonical chiral boundary of the chiral topological order $\Mod_{\cV_0}$.

For the diagonal CFT, the Hilbert space is the diagonal Lagrangian algebra
\[
L = \bigoplus_{i\in\Irr(\Mod_{\cV_0})}i\bt \bar{i}\in \Mod{}_{\cV_0}\bt \overline{\Mod{}_{\cV_0}}.
\]

\subsection{Dynamical data}
In CFT, encoding the dynamical data is alreadily achieved by the conformal invariance, as the Hamiltonian (generator of time translation) is a special component of the conformal transformation.

We focus on the VOA $\cV_0$. There exists a map called the intertwining operator $Y$\footnote{This map is usually called the state-operator correspondance in physics books. However, physics books always neglect that $Y$ is an axiom that cannot be derived.}:
\[
Y:\cV_0\rightarrow \End(\cV_0)[[z,z^{-1}]],
\]
which maps a state $\phi \in \cV_0$ to a chiral field $\phi(z):\cV_0\rightarrow \cV_0$. Here $V[[z,z^{-1}]]$ denotes the vector space of Laurant polynomial with coefficients in $V$.
\[
V[[z,z^{-1}]] = \left\{\sum_{n\in \Z}v_nz^{n}:v_n\in V\right\}.
\]

[TODO: fill the gap between the Virasoro algebra and intertwining operator by refering to \cite{vertex_operator_algebra:Wikipedia}.]

The definition of the VOA is precise, abstract but hollow. Since it is difficult to give any explicit examples, we will not dive into its detail so far.

By definition of VOA, there is a specific $\omega\in V_{-2}$, such that $T(z) = Y(\omega, z)$ is the energy-momentum tensor, which is expanded as
\[
Y(\omega, z) = \sum_{n\in\Z}L_n z^{-n-2}.
\]

\subsection{Operator product expansion}
We firstly consider a trivial case
\[
Y(\omega, z)\phi_h = \sum_{n\in\Z}z^{-n-2}L_n\phi_h = \sum_{n\leq 0}z^{-n-2}L_n\phi_h = \frac{h\phi_h}{z^2}+\frac{L_{-1}\phi_h}{z}+O(1).
\]
By definition of the vertex operator and VOA \cite{vertex_operator_algebra:Wikipedia}, 
\[
L_{-1}\phi_h = \lim_{z\rightarrow 0}L_{-1}Y(\phi_h, z)\one = \lim_{z\rightarrow 0}[L_{-1},Y(\phi_h, z)]\one = \lim_{z\rightarrow 0}\frac{\dd}{\dd z}Y(\phi_h, z)\one := \partial \phi_h.
\]
Thus the above action becomes the one that is usually seen in physics books
\[
T(z)\phi_h \sim \frac{h\phi_h}{z^2}+\frac{\partial \phi_h}{z}
\]


We define \cite{Moore:1988qv} the space of intertwining operators to be the space spanned by \footnote{Here we cheated, the right hand side is not the integer expansion.}
\[
Y:\cV_i\rightarrow \oplus_{k}\Hom(\cV_{j},\cV_{k})[[z,z^{-1}]]
\]
satisfying the quasi-conformality:
\begin{equation}\label{eq: quasiconf}
    [L_n, Y(\phi, z)] = \left((n+1)hz^n + z^{n+1}\frac{\dd}{\dd z}\right)Y(\phi, z),\quad \forall n\geq 0,\quad \phi\in V_{i}.
\end{equation}
Here $\phi$ is a primary field. 

This operator is similar to the 3j symbol in the group representation theory, where the solution of the intertwining equation gives the fusion space. The operator $Y$ is what we call operator product expansion. In the terribly written classic CFT textbook \cite{Francesco1997}, the equation \eqref{eq: quasiconf} is implicitly used to ``derive'' the general OPE, infact, \eqref{eq: quasiconf} is the only equation used there. However, it is a definition rather than a derivation. This is similar to the ``comultiplication'' map of Hopf algebras \cite{Moore:1988qv}, which is an input data to define the tensor product of two Hopf algebra modules.

We thus expand the $Y(\phi_i, z)\phi_j$. 
\[
Y(\phi_i, z)\phi_j = \sum_{p, \{N_k\in\Z\}_{k}}C_{ij}{}^{p,\{N\}}z^{\alpha_{ijp,{N}}}L_{-1}^{N_1}L_{-2}^{N_2}\cdots L_{-k}^{N_k}\phi_p.
\]
Here $\alpha_{ijpN}$ and $C_{ij}{}^{p,\{N\}}$ are numbers to be determined.

Taking $n = 0$, we have, on one hand,
\[
L_0 Y(\phi_i, z)\phi_j = \sum_{p,\{N\}}(h_p+N1+2N2+\cdots kN_k)C_{ij}{}^{p,\{N\}}z^{\alpha_{ijp\{N\}}}L_{-1}^{N_1}L_{-2}^{N_2}\cdots L_{-k}^{N_k}\phi_p.
\]
On the other hand, 
\[
L_0 Y(\phi_i, z)\phi_j = \left(h_i+h_j+z\frac{\dd}{\dd z}\right)Y(\phi_i,z)\phi_j.
\]
To make two results match, the power of $x$ must match, thus
\[
\alpha_{ijpN} = h_p-h_i-h_j+\sum_{k}kN_k.
\]

Thus we update our ansatz:
\[
Y(\phi_i, z)\phi_j = \sum_{p, \{N_k\in\N\}_{k}}C_{ij}{}^{p,\{N\}}z^{h_p-h_i-h_j+\sum_{k}kN_k}L_{-1}^{N_1}L_{-2}^{N_2}\cdots L_{-k}^{N_k}\phi_p.
\]

The next step is to successively use \eqref{eq: quasiconf}. 

As an example, we now calculate the case $C_{ij}{}^{p,\{1,0,0,\cdots\}}$. Acting $L_1$ to both hand sides, we have on one hand, 
\[
\left(2h_i z + z^2\frac{\dd}{\dd z}\right)\left(C_{ij}{}^p z^{h_p-h_i-h_j}\phi_p+C_{ij}{}^{p,\{1,0,\cdots\}}z^{h_p-h_i-h_j+1}L_{-1}\phi_p + \cdots\right).
\]
On the other hand,
\[
L_1\left(C_{ij}{}^p z^{h_p-h_i-h_j}\phi_p+C_{ij}{}^{p,\{1,0,\cdots\}}z^{h_p-h_i-h_j+1}L_{-1}\phi_p + \cdots\right) = \left(C_{ij}{}^{p,\{1,0,\cdots\}}z^{h_p-h_i-h_j+1}2h_p\phi_p + \cdots\right).
\]
By comparing the coefficient, we have
\[
C_{ij}{}^{p,\{1,0,\cdots\}} = \left(\frac{1}{2} + \frac{h_i-h_j}{2h_p}\right) C_{ij}{}^{p}.
\]

Other coefficients can be obtained recursively. Since all coefficients are determined iteratively, they all contain a term $C_{ij}{}^k$, which can be chosen freely and thus define a fusion space.

\section{Minimal model}
\subsection{Verma module}
Each $\cV_h$ is a module of $\cV_0$. Since Virasoro algebra $\mathrm{Vir} = \la T\ra\subset \cV_0$ is a subalgebra, each $\cV_h$ is at least a module of $\mathrm{Vir}$. There exists a lowest weight representation, where
\[
L_0\phi_{c,h} = h\phi_{c,h},\quad C\phi_{c,h} = c\phi_{c,h},\quad L_{n}\phi_h = 0,\quad \forall n\geq 1.
\]
For the simplicity of notation, we drop the $c$ notation in the following. The space spanned by all
\[
L_{-1}^{n_1}L_{-2}^{n_2}\cdots L_{-k}^{n_k}\phi_{h}
\]
is a module of $\mathrm{Vir}$, called the Verma module, denoted by $U(\mathrm{Vir}^-)\otimes \phi_h$\footnote{There exists a strict definition of Verma module, see \cite{Wikipedia:Verma_module}.}.


We define the subspace
\[
K = \left\{\chi\in  U(\mathrm{Vir}^-)\otimes \phi_h: U(\mathrm{Vir}^+)\cdot \chi\cap \C\phi_h = 0\right\}
\]
called the null subspace. Then the quotient space $U(\mathrm{Vir}^-)\otimes \phi_h/K$ is another module of $\mathrm{Vir}$ but with the zero modes removed.

\subsection{Kac determinant}
Let $\phi_h$ be the lowest weight vector. The minimal model aims to calculate when the Verma module $U(\mathrm{Vir}^-)\otimes \phi_h$ is ``maximally reducible''. 

Since $U(\mathrm{Vir}^-)\otimes \phi_h = \oplus_{n}V_n$ is $\Z$-graded by the eigenvalues of $L_0$, spaces with different gradings are orthogonal\footnote{Here we cheated. Previously we do not discuss star structure of VOA. In the following, we assume $L_n^\dagger = L_{-n}$.}. For each graded subspace $V_n$, we obtain a metric parametrized by $(c,h)$, called the Gram matrix, denoted as $M_n(c,h)$. The determinant of $M_n(c,h)$ is called the Kac determinant, which has been calculated explicitly
\[
\det{M_n(c,h)} = \alpha_n \prod_{r,s\geq 1}^{rs\leq n}\left(h-h_{r,s}(c)\right)^{p(n-rs)},
\]
where $p(n-rs)$ is the number of partition of integers $n-rs$, and $\alpha_n$ is the positive constant independent of $c$ and $h$. Here
\begin{align*}
    h_{r,s}(c) &= h_0 + \frac{1}{4}\left(r\alpha_++s\alpha_-\right)^2,\\
    h_0 &= \frac{1}{24}(c-1),\\
    \alpha_\pm &= \frac{\sqrt{1-c}\pm \sqrt{25-c}}{\sqrt{24}}.
\end{align*}

Thus those $h_{r,s}(c)$ are desired scaling dimensions that makes the free module maximally reducible.

We also wish the number of different $h_{r,s}(c)$ to be finite. We interprete the term $(r\alpha_++s\alpha_-)$ term as the inner product $(r,s)\cdot (\alpha_+,\alpha_-)$. Thus $(r\alpha_++s\alpha_-)^2 = \delta^2(\alpha_+^2+\alpha_-^2)$, where $\delta$ is the distance between the point $(r,s)$ and the line $\alpha_+ x+ \alpha_- y = 0$. Clearly, except that when $\tan\theta = -\frac{\alpha_+}{\alpha_-} = \frac{p}{p'}$ is rational, there are infinite points $(r,s)$ that can be arbitrarily near to the line $\alpha_+ x+ \alpha_- y = 0$.

From the condition, we can solve for $c$ in terms of $\frac{p}{p'}$:
\[
c = 13 - 6\times \frac{p^2+p'^2}{pp'} = 1- 6\times \frac{(p-p')^2}{pp'}.
\]
And
\[
h_{r,s} = \frac{(pr-p's)^2 - (p-p')^2}{4pp'},\quad 1\leq r \leq p'-1,\quad 1\leq s \leq p-1.
\]

\subsection{Reduction of the Verma module}
We now reduce the Verma module $V_{r,s}$ for each $(r,s)$ by quotienting its null subspace. We repeat the formulation in \cite{Francesco1997}, but with precise language.

For Verma module $V_{r,s}$, by the form of Kac determinant, the first null vector appears at the level $n = rs$, with scaling dimension $h_{r,s}+rs$. By the symmetricity $h_{p'-r,p-s} = h_{r,s}$, there is another null vector at the level $(p'-r)(p-s)$, with scaling dimension $h_{r,s}+(p'-r)(p-s)$. Moreover, since $p(0) = 1$, the null spaces in these ranks are all of dimension $1$. It can be proved that
\begin{align*}
    h_{r,s}+rs &= h_{p'+r,p-s},\\
    h_{r,s}+(p'-r)(p-s) &= h_{r,2p-s}.
\end{align*}
Thus as a lowest-weight vector, their Verma module are isomorphic to $V_{p'+r,p-s}$ and $V_{r,2p-s}$ respectively.

However, these two Verma modules, when embedded to $V_{r,s}$, has non-trivial union, which is the pull-back of these embeddings
% https://q.uiver.app/#q=WzAsNCxbMSwxLCJWX3tyLHN9Il0sWzAsMSwiVl97cCcrcixwLXN9Il0sWzEsMCwiVl97MnAnLXIsc30iXSxbMCwwLCJWX3twJytyLHAtc31cXHNxbltWX3tyLHN9XVZfezJwJy1yLHN9Il0sWzMsMSwiIiwwLHsic3R5bGUiOnsidGFpbCI6eyJuYW1lIjoiaG9vayIsInNpZGUiOiJ0b3AifX19XSxbMywyLCIiLDIseyJzdHlsZSI6eyJ0YWlsIjp7Im5hbWUiOiJob29rIiwic2lkZSI6InRvcCJ9fX1dLFsxLDAsIiIsMSx7InN0eWxlIjp7InRhaWwiOnsibmFtZSI6Imhvb2siLCJzaWRlIjoidG9wIn19fV0sWzIsMCwiIiwxLHsic3R5bGUiOnsidGFpbCI6eyJuYW1lIjoiaG9vayIsInNpZGUiOiJ0b3AifX19XSxbMywwLCIiLDEseyJzdHlsZSI6eyJuYW1lIjoiY29ybmVyLWludmVyc2UifX1dXQ==
\[\begin{tikzcd}
	{V_{p'+r,p-s}\xt[V_{r,s}]V_{2p'-r,s}} & {V_{2p'-r,s}} \\
	{V_{p'+r,p-s}} & {V_{r,s}}
	\arrow[hook, from=1-1, to=1-2]
	\arrow[hook, from=1-1, to=2-1]
	\arrow["\ulcorner"{anchor=center, pos=0.125}, draw=none, from=1-1, to=2-2]
	\arrow[hook, from=1-2, to=2-2]
	\arrow[hook, from=2-1, to=2-2]
\end{tikzcd}\]
Let $X :=V_{p'+r,p-s}\xt[V_{r,s}]V_{2p'-r,s}$, then the reduced space will be the quotient space of the push-forward $V_{r,s}/V_{p'+r,p-s}\squ[X]V_{2p'-r,s}$.

Here is a question that I cannot answer at the moment:
\begin{question}
    Whether the space $V_{p'+r,p-s}\squ[X]V_{2p'-r,s}$ is the total null subspace?
\end{question}

Such kind of embeddings can iteratively continue\footnote{In what sense does this embedding net ``converge''?}.


\section{Boundary CFT}
\subsection{Physical constraint}
In this section, we review the boundary CFT.

Let's suppose the CFT is defined on the upper half plane $\mathbb{H}$. Operators can either insert on $\R$ or on the upper half plane.

Let's input some physical arguments. We require that the boundary cannot change the total parallel momentum. This means\footnote{The physical meaning of $T_{yx}$ is the current flowing to the $y$-direction carrying the momentum of the $x$-direction.} \cite{cardy2008boundaryconformalfieldtheory}
\[
T_{yx}(x) = 0,\quad \forall x\in \R.
\]
Change to the holomorphic(anti-holomorphic) basis, the no-flow condition implies that
\[
T(x) := \lim_{z\rightarrow x}T(z) = \lim_{\bar{z}\rightarrow x}\overline{T}(\bar{z}),\quad \forall x\in \R.
\]

Let's be more careful of the condition here. Usually when we say that the total Virasoro algebra is the tensor product of two chiral one Virasoro algebras with the same chiral central charge:
\begin{align*}
    [L_m,\ L_n] &= (m-n)L_{m+n} + \frac{c}{12}m(m^2-1)\delta_{m+n, 0},\\
    [L'_m,\ L'_n] &= (m-n)L'_{m+n} + \frac{c}{12}m(m^2-1)\delta_{m+n, 0}.
\end{align*}
However, when we combine them to form a anomaly-free theory with zero chiral central charge, we need to flip the orientation and identify
\[
\overline{L}_n = L'_{-m},\quad [x,\ y]_{\mathrm{rev}} = [y,\ x].
\]
Then the $(\{\overline{L}_n\},[-,-]_{\mathrm{rev}})$ spans the Virasoro algebra with the reversed chiral central charge:
\[
[\overline{L}_m,\ \overline{L}_n]_{\mathrm{rev}} = (m-n)\overline{L}_{m+n} - \frac{c}{12}m(m^2-1)\delta_{m+n, 0}.
\]
And $\overline{T}(\bar{z})$ is defined by
\[
\overline{T}(\bar{z}) := \sum_{n\in\Z}\overline{L}_n \bar{z}^{-n-2}.
\]

For any boundary state $\phi_B\in \cH_{\B}$, the no-flow condition reduces to\footnote{Here, we use a different convention from \cite{cardy2008boundaryconformalfieldtheory}. The $L'$ here is $\overline{L}$ in \cite{cardy2008boundaryconformalfieldtheory}. We wish by separate $\overline{L}$ and $L'$ can help us to clarify why $n$ needs to be reversed.}
\[
L_n\cdot \phi_B = \overline{L}_{n}\cdot \phi_B = L'_{-n}\cdot \phi_B.
\]

\subsection{Ishibashi state}
Here, we look for the form of $\phi_B$ in $\cH$. There is at least a choice, by choosing $\phi_B$ to be the linear expansion
\[
\phi_B = \sum_{X, X'}\alpha_{X, X'} \left(L_{X} \cdot \phi_{h}\right)\ot \left(L'_{X'} \cdot \phi_{h'}\right).
\]
Here $\phi_h$ and $\phi_{h'}$ are two primaries and $L_{X} = L_{-1}^{X_1}\cdots L_{-n}^{X_n}$, so as $L'_{X'}$.

The condition $L_0 \cdot \phi_B = L'_0\cdot \phi_B$ implies that $h = h'$ and $h+N_X = h' + N_{X'}$. It is proved that such form of the state is unique up to a constant for each $h$, called the Ishibashi state, which is denoted as $\psi_h$:
\[
\psi_h = \sum_{X}\phi_h^{X}\ot \phi_{h}^{\prime X}.
\]

\subsection{Modular invariance}
To have a consistent partition function on a annulus, we need to do input the constraint of the modular invariance. There are two different ways to interprete the partition function on annulus. The equality of these two will become the Cardy condition of the boundary CFT.

We consider the partition function on a annulus with real modular parameter $q = \bar{q} = \ee^{2\pi \delta}$ and boundary conditions $a, b$. Here the real condition is necessary to unfold a double modular parameter to a chiral one.

On one hand, suppose \footnote{This is related to the folding picture of the doubled CFT, which we will illustrate later.}
\[
\cH_{ab} = \bigoplus_{h}n_{ab}^h \cV_{h}.
\]
Then the chiral modular parameter is $1/2$ of the double modular parameter:
\[
Z_{ab}(\delta) = \sum_{h}n^{h}_{ab}\chi_h\left(\frac{\delta}{2}\right).
\]

On the other hand, suppose the boundary condition $a$ can be expanded by
\[
\phi_a = \sum_{h}A_{a, h}\psi_h.
\]
The transition amplitude can be calculated to be
\[
\left(\psi_h ,\  \ee^{-2\pi/\delta\left(L_0 - \frac{c}{24}\right)}\ee^{-2\pi/\delta\left(\overline{L}_0-\frac{c}{24}\right)} \psi_{k}\right) = \delta_{h,k}\chi_{h}\left(-\frac{2}{\delta}\right).
\]
Then the partition function is
\[
Z_{ab}(\delta) = \sum_{h}A_{a,h}A_{b,h}^* \chi_{h}\left(-\frac{2}{\delta}\right).
\]

Thus according to the modular covariance of the character
\[
\chi_h(-1/\tau) = \sum_{k}S_{hk}\chi_k(\tau),
\]
we have the identity, called the Cardy condition:
\[
\sum_{k}A_{a,k}A_{b,k}^* S_{kh} = n^h_{ab}\in \Z_{\geq 0}.
\]

\bibliography{ref}
\bibliographystyle{abbrv}
\end{document}
