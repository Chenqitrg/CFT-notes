\documentclass{article}
\usepackage{graphicx} % Required for inserting images
\usepackage[a4paper, margin=1in]{geometry}
% math
\usepackage{amsmath, amssymb, amsthm, stmaryrd, bbold}
% plot
\usepackage{tikz, pgfplots,tikz-cd}
\usetikzlibrary{calc, decorations.markings, decorations.pathmorphing,shapes}
% form
\usepackage{diagbox}
% link
\usepackage[colorlinks, citecolor=blue, linkcolor=blue, urlcolor=blue, breaklinks=true]{hyperref}
% others
\usepackage{mdframed}
\usepackage{natbib}
\allowdisplaybreaks[4]
\pgfplotsset{compat=1.18}

% --- FONT STUFF ---
\usepackage{newpxtext}

\usepackage[T1]{fontenc}

\definecolor{CUPurple}{RGB}{117,15,109}
\definecolor{CULightPurple}{RGB}{172, 111, 167}
\definecolor{CULightPurple2}{RGB}{214, 183, 211}

%%Blues
\definecolor{YaleBlue} {HTML}{00356b}
%\definecolor{YaleMidBlue} {HTML}{286dc0}
%\definecolor{YaleLightBlue} {HTML}{63aaff}

\definecolor{YaleMidBlue} {HTML}{286dc0}
\definecolor{YaleLightBlue} {HTML}{b0d4ff}
\definecolor{YaleMidGrey} {HTML}{e0dcda}
\definecolor{YaleLightGrey} {HTML}{eeeeee}

%%Green
\definecolor{ao}{rgb}{117, 15, 109}

%%Greys
\definecolor{YaleBlack} {HTML}{222222}
\definecolor{YaleDarkGrey} {HTML}{4a4a4a}
%\definecolor{YaleMidGrey} {HTML}{978d85}
%\definecolor{YaleLightGrey} {HTML}{dddddd}
\definecolor{YaleWhite} {HTML}{f9f9f9}
\definecolor{YaleLightGreen}{HTML}{AFB896}
%%Accents
\definecolor{YaleGreen} {HTML}{5f712d}
\definecolor{YaleOrange} {HTML}{bd5319}

\newcommand{\spd}[1]{(#1+1)d}
\newcommand{\mb}[1]{\mathbf{#1}}

%% Note
\newcommand{\XY}[1]{\textcolor{YaleMidBlue}{[Xinping: #1]}}
\newcommand{\XYY}[1]{\textcolor{YaleMidBlue}{[#1]}}
\newcommand{\CQ}[1]{\textcolor{YaleMidGrey}{[Chenqi: #1]}} 

% --- Theorem
\theoremstyle{definition}
\newtheorem{theorem}{Theorem}[section]
\newtheorem{corollary}[theorem]{Corollary}
\newtheorem{prop}[theorem]{Proposition}
\newtheorem{definition}[theorem]{Definition}
\newtheorem{proposition}[theorem]{Proposition}
\newtheorem{lem}[theorem]{Lemma}
\theoremstyle{remark}
\newtheorem{remark}{Remark}
\newtheorem{convention}{Convention}
\newtheorem{conjecture}{Conjecture}
\newtheorem{example}{Example}
\newtheorem{question}{Question}

% --- Macros ---

%% Fields
\newcommand{\Z}{\mathbb{Z}}
\newcommand{\R}{\mathbb{R}}
\newcommand{\C}{\mathbb{C}}
\newcommand{\bbH}{\mathbb{H}}
\newcommand{\Q}{\mathbb{Q}}
\newcommand{\M}{\mathbb{M}}
\newcommand{\N}{\mathbb{N}}
\newcommand{\ii}{\mathrm{i}}
\newcommand{\jj}{\mathrm{j}}
\newcommand{\kk}{\mathrm{k}}
\newcommand{\ee}{\mathrm{e}}

%% Category cal
\newcommand{\cA}{ {\cal A} } 
\newcommand{\cB}{ {\cal B} }
\newcommand{\cC}{ {\cal C} } 
\newcommand{\cD}{ {\cal D} } 
\newcommand{\cE}{ {\cal E} } 
\newcommand{\cF}{ {\cal F} } 
\newcommand{\cG}{ {\cal G} } 
\newcommand{\cH}{ {\cal H} } 
\newcommand{\cI}{{\cal I}}
\newcommand{\cJ}{{\cal J}}
\newcommand{\cK}{ {\cal K} } 
\newcommand{\cL}{ {\cal L} } 
\newcommand{\cM}{ {\cal M} } 
\newcommand{\cN}{ {\cal N} } 
\newcommand{\cO}{\mathcal{O}} 
\newcommand{\cP}{ {\cal P} } 
\newcommand{\cQ}{ {\cal Q} } 
\newcommand{\cR}{ {\cal R} } 
\newcommand{\cS}{ {\cal S} } 
\newcommand{\cT}{ {\cal T} } 
\newcommand{\cU}{ {\cal U} } 
\newcommand{\cV}{ {\cal V} } 
\newcommand{\cW}{ {\cal W} } 
\newcommand{\cX}{ {\cal X} } 
\newcommand{\cY}{ {\cal Y} } 
\newcommand{\cZ}{ {\cal Z} }

%% eucal
\newcommand{\eA}{\eucal{A}}
\newcommand{\eB}{\eucal{B}}
\newcommand{\eC}{\eucal{C}}
\newcommand{\eD}{\eucal{D}}
\newcommand{\eE}{\eucal{E}}
\newcommand{\eF}{\eucal{F}}
\newcommand{\eG}{\eucal{G}}
\newcommand{\eH}{\eucal{H}}
\newcommand{\eI}{\eucal{I}}
\newcommand{\eJ}{\eucal{J}}
\newcommand{\eK}{\eucal{K}}
\newcommand{\eL}{\eucal{L}}
\newcommand{\eM}{\eucal{M}}
\newcommand{\eN}{\eucal{N}}
\newcommand{\eO}{\eucal{O}}
\newcommand{\eP}{\eucal{P}}
\newcommand{\eQ}{\eucal{Q}}
\newcommand{\eR}{\eucal{R}}
\newcommand{\eS}{\eucal{S}}
\newcommand{\eT}{\eucal{T}}
\newcommand{\eU}{\eucal{U}}
\newcommand{\eV}{\eucal{V}}
\newcommand{\eW}{\eucal{W}}
\newcommand{\eX}{\eucal{X}}
\newcommand{\eY}{\eucal{Y}}
\newcommand{\eZ}{\eucal{Z}}

%% sf
\newcommand{\sA}{ {\mathsf A} } 
\newcommand{\sB}{ {\mathsf B} }
\newcommand{\sC}{ {\mathsf C} } 
\newcommand{\sD}{ {\mathsf D} } 
\newcommand{\sE}{ {\mathsf E} } 
\newcommand{\sF}{ {\mathsf F} } 
\newcommand{\sG}{ {\mathsf G} } 
\newcommand{\sH}{ {\mathsf H} } 
\newcommand{\sI}{{\mathsf I}}
\newcommand{\sJ}{{\mathsf J}}
\newcommand{\sK}{ {\mathsf K} } 
\newcommand{\sL}{ {\mathsf L} } 
\newcommand{\sM}{ {\mathsf M} } 
\newcommand{\sN}{ {\mathsf N} } 
\newcommand{\sO}{ {\mathsf O} } 
\newcommand{\sP}{ {\mathsf P} } 
\newcommand{\sQ}{ {\mathsf Q} } 
\newcommand{\sR}{ {\mathsf R} } 
\newcommand{\sS}{ {\mathsf S} } 
\newcommand{\sT}{ {\mathsf T} } 
\newcommand{\sU}{ {\mathsf U} } 
\newcommand{\sV}{ {\mathsf V} } 
\newcommand{\sW}{ {\mathsf W} } 
\newcommand{\sX}{ {\mathsf X} } 
\newcommand{\sY}{ {\mathsf Y} } 
\newcommand{\sZ}{ {\mathsf Z} }

%% frak
\newcommand{\fC}{\mathfrak{C}}
\newcommand{\fD}{\mathfrak{D}}
\newcommand{\fA}{\mathfrak{A}}
\newcommand{\fB}{\mathfrak{B}}
\newcommand{\fS}{\mathfrak{S}}
\newcommand{\fZ}{\mathfrak{Z}}

%% Category bf
\newcommand{\BB}{\mathbf{B}}
\newcommand{\Bm}{\mathbf{m}} 
\newcommand{\Bn}{\mathbf{n}} 
\newcommand{\Br}{\mathbf{r}} 
\newcommand{\Bs}{\mathbf{s}} 
\newcommand{\Bt}{\mathbf{t}} 
\newcommand{\Bu}{\mathbf{u}} 
\newcommand{\Bv}{\mathbf{v}} 
\newcommand{\Bw}{\mathbf{w}} 
\newcommand{\Bp}{\mathbf{p}}
\newcommand{\BI}{\mathbf{I}}
\newcommand{\Ve}{\mathsf{Vec}}
\newcommand{\sVec}{\mathsf{sVec}}
\newcommand{\Hilb}{\mathsf{Hilb}}
\newcommand{\sHilb}{\mathsf{sHilb}}
\newcommand{\ssHilb}{\underline{\mathbf{sHilb}}}
\newcommand{\Grp}{\mathbf{Grp}}
\newcommand{\Ab}{\mathbf{Ab}}
\newcommand{\Is}{\mathbf{Is}}
\newcommand{\one}{\mathbf{1}}
\newcommand{\zero}{\mathbf{0}}

%% Category rm
\newcommand{\Tr}{\mathrm{Tr}}
\newcommand{\tTr}{\mathrm{tTr}}
\newcommand{\ev}{\mathrm{ev}}
\newcommand{\coev}{\mathrm{coev}}
\newcommand{\Ob}{\mathop\mathrm{Ob}}
\newcommand{\Hom}{\mathrm{Hom}}
\newcommand{\End}{\mathrm{End}}
\newcommand{\categoricalEnd}{{\mathsf{End}}}
\newcommand{\Aut}{\mathrm{Aut}}
\newcommand{\Fun}{\mathsf{Fun}}
\newcommand{\rev}{\mathrm{rev}}
\newcommand{\Pen}{\mathrm{Pen}}
\newcommand{\Rep}{\mathsf{Rep}}
\newcommand{\sRep}{\mathop\mathrm{sRep}}
\newcommand{\Mod}{\mathop\mathrm{Mod}}
\newcommand{\RMod}{\mathrm{RMod}}
\newcommand{\LMod}{\mathrm{LMod}}
\newcommand{\BMod}{\mathrm{BMod}}
\newcommand{\Free}{\mathrm{Free}}
\newcommand{\id}{\mathrm{id}}
\newcommand{\im}{\mathrm{im}}
\newcommand{\TY}{\mathrm{TY}}
\newcommand{\Irr}{\mathrm{Irr}}
\newcommand{\Idem}{\mathrm{Idem}}
\newcommand{\B}{\mathrm{B}}
\newcommand{\chargef}{\mathsf{fgt}}


%% Mathematical operators
\newcommand{\dd}{\mathrm{d}}
\newcommand{\xt}{\times}
\newcommand{\ot}[1][]{\underset{#1}{\otimes}}
\newcommand{\op}[1][]{\underset{#1}{\oplus}}
\newcommand{\oop}{\bigoplus}
\newcommand{\oot}{\bigotimes}
\newcommand{\rt}[1][]{\underset{#1}{\triangleright}}
\newcommand{\lt}[1][]{\underset{#1}{\triangleleft}}
\newcommand{\bt}[1][]{\underset{#1}{\boxtimes}}
\newcommand{\ar}{\rightarrow}
\newcommand{\Ar}{\Rightarrow}
\newcommand{\inj}{\hookrightarrow}
\newcommand{\sur}{\twoheadrightarrow}
\newcommand{\cen}[2]{{#1}^\mathrm{cen}_{#2}}
\newcommand{\Cp}{\Sigma}
\newcommand{\cond}{\mathrel{\,\hspace{.75ex}\joinrel\rhook\joinrel\hspace{-.75ex}\joinrel\rightarrow}}
\newcommand{\Cond}{\mathrel{\,\joinrel\raisebox{0.25ex}{$\rhook$}\joinrel\hspace{-1ex}\joinrel\Rightarrow}}

%% Brackets
\newcommand{\la}{\langle} 
\newcommand{\ra}{\rangle}
\newcommand{\sym}{\mathrm{sym}}
\newcommand{\scr}{\mathrm{scr}}
\newcommand{\bl}[1]{{\hat{#1}}}
\newcommand{\fgt}{\mathrm{fgt}}
\newcommand{\spann}{\mathrm{span}}

%% Tikzs
\tikzstyle string=[thin,postaction={decorate},decoration={markings,
    mark=at position 0.5*\pgfdecoratedpathlength+.5*4pt with {\arrow[scale=1]{stealth}}}]
\newcommand{\diagram}[2]{
\begin{tikzpicture}[baseline=(current bounding box),scale=#1]
#2
\end{tikzpicture}
}

% HW
\newcommand{\pt}[1]{{\color{red}(#1')}}
\newcommand{\der}[2]{\frac{\mathrm{d} #1}{\mathrm{d} #2}}
\newcommand{\pd}[2]{\frac{\partial #1}{\partial #2}}

% *** quiver ***
% A package for drawing commutative diagrams exported from https://q.uiver.app.
%
% This package is currently a wrapper around the `tikz-cd` package, importing necessary TikZ
% libraries, and defining a new TikZ style for curves of a fixed height.
%
% Version: 1.5.3
% Authors:
% - varkor (https://github.com/varkor)
% - AndréC (https://tex.stackexchange.com/users/138900/andr%C3%A9c)

\NeedsTeXFormat{LaTeX2e}
\ProvidesPackage{quiver}[2021/01/11 quiver]

% `tikz-cd` is necessary to draw commutative diagrams.
\RequirePackage{tikz-cd}
% `amssymb` is necessary for `\lrcorner` and `\ulcorner`.
\RequirePackage{amssymb}
% `calc` is necessary to draw curved arrows.
\usetikzlibrary{calc}
% `pathmorphing` is necessary to draw squiggly arrows.
\usetikzlibrary{decorations.pathmorphing}

% A TikZ style for curved arrows of a fixed height, due to AndréC.
\tikzset{curve/.style={settings={#1},to path={(\tikztostart)
    .. controls ($(\tikztostart)!\pv{pos}!(\tikztotarget)!\pv{height}!270:(\tikztotarget)$)
    and ($(\tikztostart)!1-\pv{pos}!(\tikztotarget)!\pv{height}!270:(\tikztotarget)$)
    .. (\tikztotarget)\tikztonodes}},
    settings/.code={\tikzset{quiver/.cd,#1}
        \def\pv##1{\pgfkeysvalueof{/tikz/quiver/##1}}},
    quiver/.cd,pos/.initial=0.35,height/.initial=0}

% TikZ arrowhead/tail styles.
\tikzset{tail reversed/.code={\pgfsetarrowsstart{tikzcd to}}}
\tikzset{2tail/.code={\pgfsetarrowsstart{Implies[reversed]}}}
\tikzset{2tail reversed/.code={\pgfsetarrowsstart{Implies}}}
% TikZ arrow styles.
\tikzset{no body/.style={/tikz/dash pattern=on 0 off 1mm}}

\endinput
\title{From conformal bootstrap to vertex operator algebra}
\author{Chenqi Meng}
\date{June 2025}

\begin{document}

\maketitle


\begin{abstract}
    Vertex operator algebra (VOA) clarifies which part of the CFT is \emph{definition} and which part of it is \emph{derivation}.
    This note reviews conformal field theory algebraically. We avoid the formal introduction of VOA, but always bear in mind that there is some underlying abstract definition of the CFT. 
\end{abstract}
\tableofcontents
\section{Basic ingredients}
\subsection{Kinematical data}
For closed string CFT, one considers dynamics on $S^1$. 

Usually, the Hilbert space on $S^1$ is the Lagrangian algebra in $\frak{Z}(\mathrm{Mod}_{\cV_0})$:
\[
\cH = \bigoplus_{h,\bar{h}}\cV_{h}\otimes\overline{\cV}_{\bar{h}}.
\]
Each $\cV_{h}$($\overline{\cV}_{\bar{h}}$) is naturally $\Z$-graded:
\[
\cV_h = \bigoplus_{n\in \Z} V_{h+n}.
\]
The ``ground state'' corresponds to a special vector $\one \otimes\overline{\one} \in V_{0}\otimes \overline{V}_0\subset \cV_{0}\otimes \overline{\cV}_0$.
Here $\cV_0$ is the corresponding chiral VOA, and each $\cV_h$ is a module of $\cV_0$.

\subsection{Dynamical data}
In CFT, encoding the dynamical data is alreadily achieved by the conformal invariance, as the Hamiltonian (generator of time translation) is a special component of the conformal transformation.

We focus on the chiral algebra $\cV_0$. There exists a map called the intertwining operator $Y$:
\[
Y:\cV_0\rightarrow \End(\cV_0)[[z,z^{-1}]],
\]
which maps a state $|\phi\ra\in \cV_0$ to a chiral field $\phi(z):\cV_0\rightarrow \cV_0$. Here $V[[z,z^{-1}]]$ denotes the vector space of Laurant polynomial with coefficients in $V$.
\[
V[[z,z^{-1}]] = \left\{\sum_{n\in \Z}v_nz^{n}:v_n\in V\right\}.
\]

[TODO: fill the gap between the Virasoro algebra and intertwining operator by refering to \cite{vertex_operator_algebra:Wikipedia}.]

The definition of the VOA is precise, abstract but hollow. Since it is difficult to give any explicit examples, we will not dive into its detail so far.

By definition of VOA, there is a specific $\omega\in V_{-2}$, such that $T(z) = Y(\omega, z)$ is the energy-momentum tensor, which is expanded as
\[
Y(\omega, z) = \sum_{n\in\Z}L_n z^{-n-2}.
\]

\subsection{Operator product expansion}
We firstly consider a trivial case
\[
Y(\omega, z)\phi_h = \sum_{n\in\Z}z^{-n-2}L_n\phi_h = \sum_{n\leq 0}z^{-n-2}L_n\phi_h = \frac{h\phi_h}{z^2}+\frac{L_{-1}\phi_h}{z}+O(1).
\]
By definition of the vertex operator and VOA \cite{vertex_operator_algebra:Wikipedia}, 
\[
L_{-1}\phi_h = \lim_{z\rightarrow 0}L_{-1}Y(\phi_h, z)\one = \lim_{z\rightarrow 0}[L_{-1},Y(\phi_h, z)]\one = \lim_{z\rightarrow 0}\frac{\dd}{\dd z}Y(\phi_h, z)\one := \partial \phi_h.
\]
Thus the above action becomes the one that is usually seen in physics books
\[
T(z)\phi_h \sim \frac{h\phi_h}{z^2}+\frac{\partial \phi_h}{z}
\]


Suppose there is a way to ``tensor'' $\cV_0$-modules, and there is a module intertwining operator
\[
Y:\cV_i\rightarrow \oplus_{k}\Hom(\cV_{j},\cV_{k})[[z,z^{-1}]].
\]
By the property of VOA \cite{vertex_operator_algebra:Wikipedia}\footnote{Here we cheated by considering the module intertwining operator.}, we have
\[
x^{L_0}Y(\phi_i, z)\phi_j = x^{L_0}Y(\phi_i, z) x^{-L_0} x^{L_0}\phi_j = x^{h_j}Y(x^{L_0} \phi_i, xz)\phi_j =x^{h_i + h_j}Y(\phi_i, xz)\phi_j
\]
We thus expand the $Y(\phi_i, z)\phi_j$. 
\[
Y(\phi_i, z)\phi_j = \sum_{p, \{N_k\in\Z\}_{k}}C_{ij}{}^{p,\{N\}}z^{\alpha_{ijp,{N}}}L_{-1}^{N_1}L_{-2}^{N_2}\cdots L_{-k}^{N_k}\phi_p.
\]
Here $\alpha_{ijpN}$ is a number to be determined.
We act $x^{L_0}$ to the both hand sides. On one hand, we have
\[
x^{h_i+h_j}\sum_{p,\{N\}}C_{ij}{}^{p,\{N\}}(xz)^{\alpha_{ijp{N}}}L_{-1}^{N_1}L_{-2}^{N_2}\cdots L_{-k}^{N_k}\phi_p.
\]
On the other hand, we have
\[
\sum_{p,\{N\}}C_{ij}{}^{p,\{N\}}z^{\alpha_{ijpN}}x^{h_p+\sum_{k}N_k}L_{-1}^{N_1}L_{-2}^{N_2}\cdots L_{-k}^{N_k}\phi_p.
\]
To make two results match, the power of $x$ must match, thus
\[
\alpha_{ijpN} = h_p-h_i-h_j+\sum_{k}N_k.
\]

Thus we update our ansatz:
\[
Y(\phi_i, z)\phi_j = \sum_{p, \{N_k\in\Z\}_{k}}C_{ij}{}^{p,\{N\}}z^{h_p-h_i-h_j+\sum_{k}N_k}L_{-1}^{N_1}L_{-2}^{N_2}\cdots L_{-k}^{N_k}\phi_p.
\]

The next step is to use the representation property. For simplicity, we suppose now all $\phi_{i,j,p}$ are primaries. Acting $L_n$ ($n\geq 1$) to the both hand sides, on one hand, by the quasi-conformality \cite{vertex_operator_algebra:Wikipedia}, the LHS becomes
\[
L_nY(\phi_i, z)\phi_j = [L_n, Y(\phi_i, z)]\phi_j = \left((n+1)h_iz^n + z^{n+1}\frac{\dd}{\dd z}\right)Y(\phi_i, z)\phi_j.
\]
This is similar to the ``comultiplication'' map of Hopf algebras \cite{Moore:1988qv}.

On the other hand, the RHS can be directly calculated.

We now calculate the case $C_{ij}{}^{p,\{1,0,0,\cdots\}}$. Acting $L_1$ to both hand sides, we have on one hand, 
\[
\left(2h_i z + z^2\frac{\dd}{\dd z}\right)\left(C_{ij}{}^p z^{h_p-h_i-h_j}\phi_p+C_{ij}{}^{p,\{1,0,\cdots\}}z^{h_p-h_i-h_j+1}L_{-1}\phi_p + \cdots\right).
\]
On the other hand,
\[
L_1\left(C_{ij}{}^p z^{h_p-h_i-h_j}\phi_p+C_{ij}{}^{p,\{1,0,\cdots\}}z^{h_p-h_i-h_j+1}L_{-1}\phi_p + \cdots\right) = \left(C_{ij}{}^{p,\{1,0,\cdots\}}z^{h_p-h_i-h_j+1}2h_p\phi_p + \cdots\right).
\]
By comparing the coefficient, we have
\[
C_{ij}{}^{p,\{1,0,\cdots\}} = \left(\frac{1}{2} + \frac{h_i-h_j}{2h_p}\right) C_{ij}{}^{p}.
\]

Other coefficients can be obtained recursively. Since all coefficients are determined iteratively, they all contain a term $C_{ij}{}^k$. Thus $C_{ij}{}^k$ part is the ``topological skeleton'', and the remaining part is dynamical.

\section{Minimal model}
\subsection{Verma module}
Each $\cV_h$ is a module of $\cV_0$. Since Virasoro algebra $\mathrm{Vir} = \la T\ra\subset \cV_0$ is a subalgebra, each $\cV_h$ is at least a module of $\mathrm{Vir}$. There exists a lowest weight representation, where
\[
L_0\phi_{c,h} = h\phi_{c,h},\quad C\phi_{c,h} = c\phi_{c,h},\quad L_{n}\phi_h = 0,\quad \forall n\geq 1.
\]
For the simplicity of notation, we drop the $c$ notation in the following.
We define the subspace
\[
K = \left\{\chi\in  \mathrm{Vir}\cdot \phi_h: \mathrm{Vir}\cdot \chi\cap \C\phi_h = 0\right\}
\]
called the null subspace. Then the quotient space $\mathrm{Vir}\cdot \phi_h/K$ is called the Verma module.

\subsection{Kac determinant}
Let $\phi_h$ be the lowest weight vector. The minimal model aims to calculate when the free module $\mathrm{Vir}\cdot \phi_h$ is maximally reducible. 

Since $\mathrm{Vir}\cdot \phi_h = \oplus_{n}V_n$ is $\Z$-graded by eigenvalues of $L_0$, spaces with different gradings are orthogonal\footnote{Here we cheated. Previously we do not discuss star structure of VOA. In the following, we assume $L_n^\dagger = L_{-n}$.}. For each graded subspace $V_n$, we obtain a metric parametrized by $(c,h)$, called the Gram matrix, denoted as $M_n(c,h)$. The determinant of $M_n(c,h)$ is called the Kac determinant, which has been calculated explicitly
\[
\det{M_n(c,h)} = \alpha_n \prod_{r,s\geq 1}^{rs\leq n}\left(h-h_{r,s}(c)\right)^{p(n-rs)},
\]
where $p(n-rs)$ is the number of partition of integers $n-rs$, and $\alpha_n$ is the positive constant independent of $c$ and $h$. Here
\begin{align*}
    h_{r,s}(c) &= h_0 + \frac{1}{4}\left(r\alpha_++s\alpha_-\right)^2,\\
    h_0 &= \frac{1}{24}(c-1),\\
    \alpha_\pm &= \frac{\sqrt{1-c}\pm \sqrt{25-c}}{\sqrt{24}}.
\end{align*}

Thus those $h_{r,s}(c)$ are desired scaling dimensions that makes the free module maximally reducible.

We also wish the number of different $h_{r,s}(c)$ to be finite. We interprete the 
\bibliography{ref}
\bibliographystyle{abbrv}
\end{document}
